% This file contains the layout of the first six pages and the explanation text
% of a compilation thesis. 

%%%%%%%%%%%%% FIRST PAGES WITH THESIS TITLE ETC %%%%%%%%%%%%%%%%%%%%%%%%%%%%%%%%
% \frontmatter % roman numbers

% Page one: Title only
\thispagestyle{empty} % no page number
\begin{center}
\vspace*{5cm}
{\Large \myMainTitle}
%\\
%{\large \mySubTitle}
\end{center}

% Page two: empty. Empty space is important

%%%%%%%%%%%%%%%%%%%%%%%%%%%%%%%%%%%%%%%%%%%%%%%%%%%%%%%%%%%%%%%%%%%%%%%
% Page three: title page with small text (spikningsblad)

\cleardoublepage
\thispagestyle{empty} % no page number
~
\vfill
\begin{center}
{\HUGE \myMainTitle}
\\[2mm]
{\huge \mySubTitle}

\vfill
{by \myName}

\vfill
% black and white (default):
\includegraphics[width=0.25\textwidth]{LundUniversity_C2line_BLACK.eps}

\vspace{10mm}
{\large \myDegree}\\
{\large Thesis advisors: \myAdvisors}\\
{\large Faculty opponent: \myOpponent}\\
\vspace{1cm}
{\footnotesize
\myDefenceAnnouncement
}
\\
\end{center}
\vfill

%%%%%%%%%%%%%%%%%%%%%%%%%%%%%%%%%%%%%%%%%%%%%%%%%%%%%%%%%%%%%%%%%%%%%%%
% Page four: data sheet
\newpage \thispagestyle{empty} % no page number
%Either include datasheet texfile (will be filled in automatically), or a pdf
%containing datasheet (text needs to be already in it, edit
%sheetPDF_editable.pdf). Use one of the next two lines: 
\input{auxiliary_texfiles/datasheet}
%\addtocounter{pages}{1} \includepdf[pages=1-1]{datasheetPDF_editable}

%%%%%%%%%%%%%%%%%%%%%%%%%%%%%%%%%%%%%%%%%%%%%%%%%%%%%%%%%%%%%%%%%%%%%%%
% Page five: title and author, without small text. Looks good!

\cleardoublepage
\thispagestyle{empty} % no page number
~
\vfill
\begin{center}
{\HUGE \myMainTitle}
\\[2mm]
{\huge \mySubTitle}

\vfill
{by \myName}

\vfill
% black and white (default):
\includegraphics[width=0.25\textwidth]{LundUniversity_C2line_BLACK.eps}

% Colour text in white so that the spacing is the same as on page three, but with less clutter
\color{white}{
\vspace{10mm}
{\large \myDegree}\\
{\large Thesis advisors: \myAdvisors}\\
{\large Faculty opponent: \myOpponent}\\
\vspace{1cm}
{\footnotesize
\myDefenceAnnouncement
}
}
\\
\end{center}
\vfill


%%%%%%%%%%%%%%%%%%%%%%%%%%%%%%%%%%%%%%%%%%%%%%%%%%%%%%%%%%%%%%%%%%%%%%%
% Page six: Cover image description, ISBN, copyright
\newpage 
\thispagestyle{empty} % no page number
%~
%\vfill
\vspace{-15mm}
A doctoral thesis at a university in Sweden takes either the form of a single,
cohesive research study (monograph) or a summary of research papers
(compilation thesis), which the doctoral student has written alone or together
with one or several other author(s). 

In the latter case the thesis consists of two parts. An introductory text puts
the research work into context and summarizes the main points of the papers.
Then, the research publications themselves are reproduced, together
with a description of the individual contributions of the authors. The
research papers may either have been already published or are manuscripts at
various stages (in press, submitted, or in draft). 

\vfill
{\small
\myCoverFront\\
\\
\myCoverBack\\
\\
\myFundingInformation


\vspace{5mm}
\copyright\, \myName~\myYear\\
\\
\myFaculty, {\myDepartment}
\\
\\
\ISBN: \myISBNprint~(print)\\ % ISBN av svenska ISBN centralen
\ISBN: \myISBNpdf~(pdf)\\ % ISBN av svenska ISBN centralen
\mySeries\\
\\
Printed in Sweden by Media-Tryck, Lund~University, Lund~\myYear

\includegraphics[width=0.5\textwidth]{ENG-Miljologotyper-sid-2-BLACK.eps}
}
